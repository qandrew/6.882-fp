\documentclass[runningheads]{article}
\usepackage[utf8]{inputenc}

\title{6.882 Proposal: Bayesian Inference on Popularity}
\author{Andrew Xia, Kelvin Lu, Brandon Zeng}
\date{March 23 2018}

\begin{document}

\maketitle

\section{Project Proposal}

For our final project, we are interested in using Bayesian Analysis techniques to better understand how to predict tweet popularity from existing Twitter data. 

We plan on implementing a paper by Zaman et al \cite{popularity} which details a Bayesian approach for predicting the popularity of tweets, in which the popularity of a tweet is defined as the number of retweets it receives. The paper analyzes a tweet and examines properties of the retweet graph, such as the depth, the time delay between retweets, in addition to properties of the user (number of followers, etc) to create a generative model of the probability that a tweet will be retweeted. 

Our main priority is to replicate the model with the data provided in \cite{popularity}. In addition, the dataset in \cite{medialab} contains over 120,000 tweets and their subsequent retweet graph, classified on whether the original tweet was "real" or "fake" news\footnote{Fake news is a feature that we initially won't be examining but it could also be an interesting feature to encorporate!}.

% The paper uses MCMC

If time permits, we also hope to augment the model presented in \cite{popularity} by including additional features such as the time of day of the tweet. One deficiency in \cite{popularity} is that the model doesn't incorporate the actual text in the tweet as a feature. We hope to include text as a feature to our model, using models such as TF-IDF or word2vec. Finally, analyzing the popularity and spread of tweets may not be based on a single tweet; in fact, multiple tweets (such as trending topics) may affect the spread of a tweet. We can look at hashtags within a tweet to determine correlation between tweets in our model.

\section{Plan and Deadlines}

% \subsection{Steps}

We plan to take the following steps to complete our project, as detailed in the following timeline:

% Firstly, we will

% Next, we will

% Finally, we will

% \subsection{Timeline}

\begin{enumerate}
    \item Week of 3/26 (Spring Break): we plan on spending more time conducting background reading and exploring more papers to impwlement.
    \item Week of 4/2 Complete explaratory data analysis to familiarize ourselves with the dataset (i.e., tweets, retweet graphs, reaction times). Implement generative model for retweet graph evolution.
    \item Week of 4/9: Submit Progress Report. Implement log-normal model for reaction times and bionomial model for retweet graph structure. 
    \item Week of 4/16 Combine the models into an overall graphical log-normal-binomial model for the evolution of retweet graphs.
    \item Week of 4/23 Implement the posterior distribution, and sample using MCMC.
    \item Week of 4/30 Evaluate our model using absolute percent error between predicted and actual tweets. As a baseline, we can compare against a regression model that uses tweet counts.
    \item Week of 5/7: Write the project paper and submit.
    
\end{enumerate}

\section{Division of Labour}

For this project, we will be working on separate portions of the model concurrently (e.g. one person will implement the log-normal portions of the model and another person will implement the binomial model portions). All of our code will be available and public at \texttt{https://github.com/qandrew/6.882-fp}. Data collection, sampling, and writing the final report will be done collaboratively.

\section{Project Risks}

One major risk to the progress of our project is our speed of implementation. If implementing the hierarchical graphical model becomes time-consuming, we will consult references online on implementing MCMC samplers, etc.

While we have already located our data, finding additional data to test our model may also become problematic. The datasets from \cite{medialab, popularity} are formatted differently, so data processing may be time consuming. If we are sparse on time, we will focus on implementing our model on the data from \cite{popularity} first.

\nocite{*}
\bibliographystyle{amsplain}
\bibliography{papers}

\section*{Appendix}

\subsection*{Pre-Proposal}

We initially focused on applying Bayesian inference on the spread of fake news on Twitter data, in which our current project is a generalized approach to this problem space. Our pre-proposal is available below for reference (although not directly relevant).

We hope to build upon previous work in this space in our analysis. For example, the Bayesian Echo Chamber \cite{bayes_echo} uses a model that when two people interact, the first person's use of certain words can increase the second person’s probability of subsequently using such word. This in turn can create a model on a social graph to determine which accounts carry more influence, and help us propagate our detection of fake accounts. Another vein of work we can consider, influenced by a recent paper on fake tweets by the Media Lab \cite{medialab}, is to infer the validity of a tweet based on the retweet structure (who retweets, how often, etc.).

\end{document}

For the proposal, I'll be looking for a lot more concreteness. Will
you be replicating these papers? If so, exactly what are the steps you
will take to do so? What models, inference techniques, and evaluations
will you conduct? What data will you use?

% Notes
% * Develop a clear timeline with intermediate, well-defined steps for each individual in the group
% * Be specific about how you will validate your work (how will you check any code vs your model choices applied to particular data)

% References
% https://web.stanford.edu/class/cs221/2017/restricted/p-final/arjunvb/final.pdf
% http://proceedings.mlr.press/v38/guo15.pdf bayesian echo chamber



