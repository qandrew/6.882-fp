\documentclass{article}
\usepackage[utf8]{inputenc}

\title{6.882 Proposal: Bayesian Inference on Fake News}
\author{Andrew Xia, Kelvin Lu, Brandon Zeng}
\date{March 16 2018}

\begin{document}

\maketitle

\section{Project Pre-Proposal}

For our final project, we are interested in using Bayesian Analysis techniques to better understand how to classify fake news on Twitter data. 

We hope to build upon previous work in this space in our analysis. For example, the Bayesian Echo Chamber \cite{bayes_echo} uses a model that when two people interact, the first person's use of certain words can increase the second person’s probability of subsequently using such word. This in turn can create a model on a social graph to determine which accounts carry more influence, and help us propagate our detection of fake accounts. Another vein of work we can consider, influenced by a recent paper on fake tweets by the Media Lab \cite{medialab}, is to infer the validity of a tweet based on the retweet structure (who retweets, how often, etc.).

\nocite{*}
\bibliographystyle{amsplain}
\bibliography{papers}

\end{document}

% References
% https://web.stanford.edu/class/cs221/2017/restricted/p-final/arjunvb/final.pdf
% http://proceedings.mlr.press/v38/guo15.pdf bayesian echo chamber


